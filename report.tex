\documentclass[a4paper,12pt,titlepage,finall]{article}

\usepackage[T1,T2A]{fontenc}     % форматы шрифтов
\usepackage[utf8x]{inputenc}     % кодировка символов, используемая в данном файле
\usepackage[russian]{babel}      % пакет русификации
\usepackage{tikz}                % для создания иллюстраций
\usepackage{pgfplots}            % для вывода графиков функций
\usepackage{geometry}		 % для настройки размера полей
\usepackage{indentfirst}         % для отступа в первом абзаце секции
\usepackage{multirow}            % для таблицы с результатами
\usepackage{amsmath}             % для системы уравнений
\usepackage{courier}             % шрифт для кода

% выбираем размер листа А4, все поля ставим по 3см
\geometry{a4paper,left=30mm,top=30mm,bottom=30mm,right=30mm}

\setcounter{secnumdepth}{0}      % отключаем нумерацию секций

\usepgfplotslibrary{fillbetween} % для изображения областей на графиках

\begin{document}
% Титульный лист
\begin{titlepage}
    \begin{center}
	{\small \sc Московский государственный университет \\имени М.~В.~Ломоносова\\
	Факультет вычислительной математики и кибернетики\\}
	\vfill
	{\Large \sc Отчет по заданию №1}\\
	~\\
	{\large \bf <<Методы сортировки>>}\\ 
	~\\
	{\large \bf Вариант 3 / 4 / 1 / 3}
    \end{center}
    \begin{flushright}
	\vfill {Выполнил:\\
	студент 102 группы\\
	Шутков~Г.~А.\\
	~\\
	Преподаватель:\\
	Кулагин~А.~В.}
    \end{flushright}
    \begin{center}
	\vfill
	{\small Москва\\2020}
    \end{center}
\end{titlepage}

% Автоматически генерируем оглавление на отдельной странице
\tableofcontents
    
\newpage

\section{Постановка задачи}

Требуется реализовать два метода сортировки: "пузырьком" и сортировку Шелла. Затем необходимо
сравнить их эффективность на тестовом наборе данных и произвести теоретическую оценку
сложности работы алгоритмов. Входными данными являются десятичные числа с плавающей точкой, 
а сортировка происходит по невозрастанию модулей элементов.

\newpage

\section{Результаты экспериментов}

\textbf{Сортировка пузырьком}

\begin{table}[h]
\centering
\begin{tabular}{|c|c|c|c|c|c|c|c|}
    \hline
    \multirow{2}{*}{\textbf{n}} & \multirow{2}{*}{\textbf{Параметр}} & \multicolumn{4}{|c|}{\textbf{Номер сгенерированного массива}} & \textbf{Среднее} \\
    \cline{3-6}
    & & \parbox{1.5cm}{\centering 1} & \parbox{1.5cm}{\centering 2} & \parbox{1.5cm}{\centering 3} & \parbox{1.5cm}{\centering 4} & \textbf{значение} \\
    \hline
    \multirow{2}{*}{10} & Сравнения & 45 & 45 & 45 & 45 & 45 \\
    \cline{2-7}
                        & Перемещения & 9 & 0 & 36 & 21 & 16.5 \\
    \hline
    \multirow{2}{*}{100} & Сравнения & 4095 & 4095 & 4095 & 4095 & 4095 \\
    \cline{2-7}
                         & Перемещения & 4095 & 0 & 2498 & 2599 & 2298\\
    \hline
    \multirow{2}{*}{1000} & Сравнения & 499500 & 499500 & 499500 & 499500 & 499500 \\
    \cline{2-7}
                          & Перемещения & 499500 & 0 & 250170 & 245301 & 248742\\
    \hline
    \multirow{2}{*}{10000} & Сравнения & 49995000 & 49995000 & 49995000 & 49995000 & 49995000 \\
    \cline{2-7}
                           & Перемещения & 49995000 & 0 & 25185157 & 25017928 & 25049521 \\
    \hline
\end{tabular}
\caption{Результаты работы сортировки пузырьком}
\end{table}

\textbf{Оценка сложности алгоритма сортировки пузырьком}

Данная сортировка совершает последовательные проходы от начала массива до его конца и
производит поочередное сравнение соседних элементов. Элементы меняются местами,
если их порядок не соответствует выбранному. Так как в алгоритме меняться местами могут только соседние элементы, 
то каждый обмен уменьшает количество инверсий на единицу. Следовательно, количество обменов равно 
количеству инверсий в исходном массиве вне зависимости от реализации сортировки. 
Максимальное количество инверсий содержится в массиве, элементы которого отсортированы в обратном порядке, 
их в нем $\frac{n(n−1)}{2}$. Получаем, что в худшем случае ассимптотика алгоритма составит $O(n^2)$. 
Лучшим случаем является отсортированный массив и в зависимости от реализиции ассимптотика составит $O(n)$ или $O(n^2)$.
В среднем этот алгоритм сортирует произвольный массив за $O(n^2)$, поскольку в нем $\frac{n(n-1)}{4}$ инверсий.

\newpage

\textbf{Сортировка Шелла}

\begin{table}[h]
\centering
\begin{tabular}{|c|c|c|c|c|c|c|c|}
    \hline
    \multirow{2}{*}{\textbf{n}} & \multirow{2}{*}{\textbf{Параметр}} & \multicolumn{4}{|c|}{\textbf{Номер сгенерированного массива}} & \textbf{Среднее} \\
    \cline{3-6}
    & & \parbox{1.5cm}{\centering 1} & \parbox{1.5cm}{\centering 2} & \parbox{1.5cm}{\centering 3} & \parbox{1.5cm}{\centering 4} & \textbf{значение} \\
    \hline
    \multirow{2}{*}{10} & Сравнения & 45 & 9 & 28 & 13 & 24 \\
    \cline{2-7}
                        & Перемещения & 45 & 0 & 19 & 4 & 19.25 \\
    \hline
    \multirow{2}{*}{100} & Сравнения & 620 & 275 & 753 & 711 & 590 \\
    \cline{2-7}
                         & Перемещения & 430 & 0 & 511 & 471 & 353 \\
    \hline
    \multirow{2}{*}{1000} & Сравнения & 9217 & 5616 & 13229 & 13419 & 10370 \\
    \cline{2-7}
                          & Перемещения & 4478 & 0 & 7977 & 8175 & 5158 \\
    \hline
    \multirow{2}{*}{10000} & Сравнения & 132571 & 86021 & 197172 & 195560 & 152831 \\
    \cline{2-7}
                           & Перемещения & 55174 & 0 & 115024 & 113436 & 70909 \\
    \hline
\end{tabular}
\caption{Результаты работы сортировки Шелла}
\end{table}

\textbf{Оценка сложности алгоритма сортировки Шелла}

Сортировка Шелла является усовершенствованной версией сортировки вставками. На каждом 
проходе алгоритма по массиву выберается велечина $L_i$, описывающая расстояние между соседними
элементами. На последнем проходе $L_0 = 1$. После определения $L_i$, массив разбивается на несколько
списков, в каждом их которых расстоние между соседними элементами равно $L_i$. Списков ровно $L_i$ штук.
Далее каждый из получившихся списков сортируется при помощи сортировки вставками. В конуе итерации
списки снова объединяются в один массив. Легко понять, что ассимптотик алгоитма зависит от способа 
выбора $L_i$ на каждом шаге. В данной реализации используется формула, полученная Робертом Седжвиком:
\begin{equation*}
L_i = 
\begin{cases}
9*(2^i-2^\frac{i}{2})+1 &\text{если i - четно}
\\
8*2^i-6*2^\frac{i + 1}{2} + 1 &\text{если i - нечетно}
\end{cases}
\end{equation*}
При таком методе выбора шага, алгоритм работает в среднем за $O(n^\frac{7}{6})$, а в худшем случае
не медленнее, чем за $O(n^\frac{4}{3})$.


\newpage

\section{Структура программы и спецификация функций}

\begin{flushleft}
\begin{itemize}

\item \texttt{int comp(const void *a, const void *b)}

Функция принимает на вход указатели на два сравниваемых элемента массива, преобразует их к типу \texttt{double} и возвращает:
\begin{equation*}
\begin{cases}
1, если |a| > |b|\\
-1, если |a| < |b|\\
0, если |a| = |b|
\end{cases}
\end{equation*}

\item \texttt{int *step\_precnt(int n)}

Функция принимает на вход одно целое число типа \texttt{int}, формирует массив сдвигов для сортировки Шелла и возвращает указатель на его начало.
\item \texttt{void shell\_sort(double *arr, int n)}

Функция принимает на вход массив чисел типа \texttt{double} и его размер, а затем сортирует его алгоритмом Шелла.
\item \texttt{void bubble\_sort(double *arr, int n)}

Функция принимает на вход массив чисел типа \texttt{double} и его размер, а затем сортирует его алгоритмом "пузырек".
\item \texttt{void gen\_arr(double *arr, int n, int k, int min, int max)}

Функция принимает указатель на генерируемый массив типа \texttt{double}, его длину, параметр, задающий упорядоченность элементов и 
границы, в которых лежат элементы этого массива.

\end{itemize}
\end{flushleft}

\newpage

\section{Отладка программы, тестирование функций}

Отладка программы производилась при помощи вывода на экран генерируемых и уже отсортированных массивов.
Проверка корректности подсчета числа сравнений и обменов делалась вручную на данных маленького объема.
Также производилось измерение времени работы сортировок для лучшего понимания их эффективности.

\newpage

\section{Анализ допущенных ошибок}

В ходе написания программы была допущена критическая ошибка, вследствии которой сортировка Шелла работала за $O(n^2)$, 
но после проведения тестов с замерением времени и сопоставления выходных данных програмы она была устранена.

\newpage
\begin{raggedright}
\addcontentsline{toc}{section}{Список цитируемой литературы}
\begin{thebibliography}{99}
\bibitem{cs} Кормен Т., Лейзерсон Ч., Ривест Р, Штайн К. Алгоритмы: построение и анализ.
    Второе издание.~--- М.:<<Вильямс>>, 2005.
\end{thebibliography}
\end{raggedright}

\end{document}
